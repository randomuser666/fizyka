\documentclass[12pt,a4paper]{article}
\usepackage[utf8]{inputenc}
\usepackage[MeX]{polski}
\usepackage{amsmath}
\usepackage{amsfonts}
\usepackage{amssymb}
\usepackage{graphicx}
\begin{document}

\section{Grupa 1}
\begin {enumerate}


\item Dipol (moment dipolowy), rysunek, wzór

\item Zależność kondensatora z dielektryka do wymiarów geometrycznych (wzór)

\item siłą Lorentza i rysunek

 $$\vec{F_B}=q\vec{v}\times\vec{B}$$
 $$F_B = qv_dBsin90^o=\frac{IL}{v_d}v_dBsin90^o$$
 $$F_B=ILB$$
 $$\vec{F_B}=I\vec{L}\times\vec{B}$$

\item Ciało doskonale czarne i Boltzmana

\item Równanie siatki dyfrakcyjnej

Siatka dyfrakcyjna to zbiór szczelin: prostoliniowych, równoległych i równoodległych. Stała siatki (d) to ilość szczelin przypadających na 1 mm. Światło ze wszystkich szczelin rozkłada się na ekranie. Obserwowany jest obraz interferencyjny jak z dwóch szczeli, ale o lepszej rozdzielczości.\\
Warunek wzmocnienia: $dsin\theta =m\uplambda$, gdzie $d$ - odległość między szczelinami, $\theta$ - kąt odchylenia

\item prawo załamania

Promień padający, promień załamany i normalna do powierzchni załamania, wystawiona w punkcie padania, leżą w jednej płaszczyźnie. Stosunek sinusa kąta padania $\alpha$ do sinusa kąta załamania $\beta$ dla dwóch ośrodków jest równy stosunkowi prędkości $v_1$ rozchodzenia się fali w pierwszym ośrodku do prędkości $v_2$ w drugim ośrodku
$$\frac{sin\alpha}{sin\beta}=\frac{v_1}{v_2}$$

\item efekt Comptona

rozpraszanie fal elektromagnetycznych na swobodnych elektronach -- doświadczalne potwierdzenie cząstkowej natury światła. Polega na pomiarze natężenia wiązki rozproszonej pod różnymi kątami $\varphi$ jako funkcja długości fali $\uplambda$
\textbf{Wyniki doświadczenia Comptona}: Promieniowanie rozproszone ma dwie składowe o długościach fali: $\uplambda = \uplambda_0$ i $\uplambda = \uplambda_0 +\Delta\uplambda$. Przesunięcie Comptona $\Delta\uplambda$ zwiększa się wraz ze wzrostem kąta rozpraszania $\varphi$\\
Przyjmując padające promieniowanie jako falę - pojawienie się fali rozproszonej o zmienionej długości fali nie daje się wyjaśnić. Przyjęcie hipotezy, że wiązka promieni X jest strumieniem fotonów o energi $hv$ pozwoliło Comptonowi wyjaśnić uzyskane wyniki\\
Z zasady zachowania energii i zasady zachowania pędu -- wzór na przesunięcie Camptonowskie :
$$\Delta\uplambda = \uplambda' -\uplambda_0 = \frac{h}{m_0c}\left(1-cos\varphi\right)$$
$m_0$ - masa elektronu (spoczynkowa)

\item stan stacjonarny, wzburzony

Stan stacjonarny atomu, w którym elektron porusza się po orbicie o najniższej energii -- stan podstawowy, dla atomu wodoru
$$E_1 = -13.6eV$$
Stan wzbudzony -- stan, w którym energia elektronu jest wyższa, żnajduje się on na wyższej orbicie
Wartości energii dozwolonych stanów stacjonarnych (skwantowane):
$$E_n = \frac{E_1}{n^2}$$
Największa (?) wartość = 0 dla $n=\infty$

\item półprzewodniki, rodzaje, opis

Podstawowe cechy:
\begin{itemize}
\item Konduktywność pomiędzy przewodnikami a izolatorami
\item Przerwa energetyczna 0.1-2eV
\item W temperatorze pokojowej występują elektrony w paśmie przewodnictwa
\item Wraz ze wzrostem temperatury rezystancja półprzewodnika maleje
\item Działając na półprzewodnik ciepłem, promieniowaniem, polami elektrycznym lub magnetycznym łatwo jest przenieść elektron z pasma podstawowego do pasma przewodnictwa
\end{itemize}

Półprzewodniki - dziury i elektrony - Zerwanie wiązania elektronowego jest równoznaczne z pojawieniem się luki w sieci wiązań międzyatomowych. \\

Przejście pomiędzy pasmami -- generacja i rekombinacja pary dziura-elektron\\

\textbf{Prąd w półprzewodniku:} \emph{elektronowy} w paśmie przewodnictwa w kierunku elektrody dodatniej, \emph{dziurowy} w paśmie podstawowym w kierunku elektrody ujemnej
\begin{itemize}
\item Ruchliwość dziur jest znacznie mniejsza od ruchliwości elektronów
\item O przewodności półprzewodnika decyduje liczba jego elektronów i dziur
\item Nośniki większościowe -- decydują o prądzie w półprzewodniku (większy wkład w przepływ prądu)
\item Nośniki mniejszościowe -- mające mniejszy wpływ na przepływ prądu przez półprzewodnik
\item W zależności od technologii wykonania nośnikami większościowymi mogą dziury lub elektrony
\end{itemize}

Przewodnictwo elektronowe (typu n) - przenoszenie łądunku elektrycznego przez ciało pod działaniem zewnętrznego pola elektrycznego. W modelu pasmowym krystalicznych ciał stałych zjawosko polegające na tym, że elektrony zajmujące stany kwantowe w obrębie pasma przewodnictwa przesuwają się do sąsiednich, nie obsadzonych stanów kwantowych w obrębie tego pasma, w kierunku przeciwnym do kierunku pola elektrycznego

Przewodnictwo dziurowe (typu p) -- przenoszenie ładunku elektrycznego przez kryształ pod działaniem zewnętrznego pola elektrycznego, polegające na tym, że elektrony pozostające w niecałkowicie zapełnionym pamie podstawowym przesuwają się do niezajętych poziomów kwantowych (dziur elektronowych) w obrębie tego pasma w kierunku przeciwnym do wektora pola elektrycznego, co formalnie odpowiada przesuwaniu się ładunków dodatnich zgodnie z kierunkiem pola elektrycznego

Materiały półprzewodnikowe:
\begin{itemize}
\item Półprzewodniki - grupa materiałów, które ze względu na przewodnictwo elektryczne zajmują pośrednie miejsce pomiędzy metalami a izolatorami.
\item W dostatecznie niskich temperaturach półprzewodnik staje się izolatorem.
\item W temperaturze zera bezwzględnego mają całkowicie obsadzone pasmo walencyjne i całkowicie puste pasmo przewodnictwa
\item Zmiana elektrycznego w wyniku niewielkich zmian ich składu
\item Dzielą się na samoistne i niesamoistne (typu p i n)
\end{itemize}

\item fala materii, długość fali de Broglie'a

\item defekt masy i energia wiązania

\item rodzaje reakcja jądra(rozpadu) i zależność z liczbą kwantową

\end{enumerate}

\section{Grupa 2}

\begin {enumerate}

\item równanie Heisenberga

\item charakterystyka ładunku elektrycznego

\item liczby kwantowe

 \begin{itemize}
 	\item Kwantyzacja właściwie wszystkich wielkości fizycznych, mierzonych w mikroświecie atomów i cząsteczek (wielkości mogą przyjmować tylko pewne ściśle określone wartości)
 	\item Elektrony w atomie znajdują się na ściśle określonych orbitach
 	\item Każdej orbicie elektronowej odpowiada pewna energia
 	\item Bliższe badania pokazały, że w podobny sposób zachowują się także inne wielkości np pęd, moment pędu czy moment magnetyczny (kwantowaniu podlega tu nie tylko wartość, ale i położenie wektora w przestrzeni albo jego rzutu na wybraną oś) $\Rightarrow$ Ponumerowanie wszystkich możliwych wartości np energii czy momentu pędu, te numery to właśnie liczby kwantowe
 \end{itemize}

\item postulaty Bohra

 \begin{itemize}
 	\item Elektron w atomie wodoru porusza się po kołowej orbicie dookoła jądra pod wpływem siły coulombowskiej i zgodnie z prawami Newtona
 	\item Elektron może poruszać się po takiej orbicie dla której moment pędu jest równy wielokrotności stałej Plancka
 	$$mv_nr_n=n\frac{h}{2\pi}$$
 	$v_n$ - prędkość elektronu na ntej orbicie, $r_n$ - promień ntej orbity, $n$ - główna liczba kwantowa, $h$ - stała Plancka
 	\item Elektron poruszający się po orbicie stacjonarnej nie wypromieniowuje energii elektromagnetycznej
 	\item Atom przechodząc ze stanu $E_n$ do stanu $E_k$ wypromieniowuje kwant energii
 	$$hv = E_n-E_k$$
 \end{itemize}

\item kąt graniczny

\item siła elektrodynamiczna

\item wzór Einsteina do czegoś fotoelektrycznego(?!)

\item Huygens - prawo, rysunek

\item zależność oporu elektrycznego od wymiarów

\item prawo Wiena

\item izotopy co to

\item rodzaje ciał stałych i wpływ przewodnictwa


\end {enumerate}
\end{document}
