\documentclass[12pt,a4paper]{article}
\usepackage[utf8]{inputenc}
\usepackage[MeX]{polski}
\usepackage{amsmath}
\usepackage{amsfonts}
\usepackage{amssymb}
\usepackage{graphicx}
\begin{document}

\section{Grupa 1}
\begin {enumerate}


\item Dipol (moment dipolowy), rysunek, wzór

\item Zależność kondensatora z dielektryka do wymiarów geometrycznych (wzór)

\item siłą Lorentza i rysunek

 $$\vec{F_B}=q\vec{v}\times\vec{B}$$
 $$F_B = qv_dBsin90^o=\frac{IL}{v_d}v_dBsin90^o$$
 $$F_B=ILB$$
 $$\vec{F_B}=I\vec{L}\times\vec{B}$$

\item Ciało doskonale czarne i Boltzmana

\item Równanie siatki dyfrakcyjnej

\item prawo załamania

\item efekt Comptona

\item stan(?) stacjonarny, wzburzony

\item półprzewodniki, rodzaje, opis

\item fala materii, długość fali de Broglie'a

\item defekt masy i energia wiązania

\item rodzaje reakcja jądra(rozpadu) i zależność z liczbą kwantową

\end{enumerate}

\section{Grupa 2}

\begin {enumerate}

\item równanie Heisenberga

\item charakterystyka ładunku elektrycznego

\item liczby kwantowe

 \begin{itemize}
 	\item Kwantyzacja właściwie wszystkich wielkości fizycznych, mierzonych w mikroświecie atomów i cząsteczek (wielkości mogą przyjmować tylko pewne ściśle określone wartości)
 	\item Elektrony w atomie znajdują się na ściśle określonych orbitach
 	\item Każdej orbicie elektronowej odpowiada pewna energia
 	\item Bliższe badania pokazały, że w podobny sposób zachowują się także inne wielkości np pęd, moment pędu czy moment magnetyczny (kwantowaniu podlega tu nie tylko wartość, ale i położenie wektora w przestrzeni albo jego rzutu na wybraną oś) $\Rightarrow$ Ponumerowanie wszystkich możliwych wartości np energii czy momentu pędu, te numery to właśnie liczby kwantowe
 \end{itemize}

\item postulaty Bohra

 \begin{itemize}
 	\item Elektron w atomie wodoru porusza się po kołowej orbicie dookoła jądra pod wpływem siły coulombowskiej i zgodnie z prawami Newtona
 	\item Elektron może poruszać się po takiej orbicie dla której moment pędu jest równy wielokrotności stałej Plancka
 	$$mv_nr_n=n\frac{h}{2\pi}$$
 	$v_n$ - prędkość elektronu na ntej orbicie, $r_n$ - promień ntej orbity, $n$ - główna liczba kwantowa, $h$ - stała Plancka
 	\item Elektron poruszający się po orbicie stacjonarnej nie wypromieniowuje energii elektromagnetycznej
 	\item Atom przechodząc ze stanu $E_n$ do stanu $E_k$ wypromieniowuje kwant energii
 	$$hv = E_n-E_k$$
 \end{itemize}

\item kąt graniczny

\item siła elektrodynamiczna

\item wzór Einsteina do czegoś fotoelektrycznego(?!)

\item Huygens - prawo, rysunek

\item zależność oporu elektrycznego od wymiarów

\item prawo Wiena

\item izotopy co to

\item rodzaje ciał stałych i wpływ przewodnictwa


\end {enumerate}
\end{document}
